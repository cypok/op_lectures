%\documentclass[xcolor=dvipsnames,12pt]{beamer}

% hide navigation symbols
\beamertemplatenavigationsymbolsempty

% add frame number
% TODO make it simplier
\makeatletter
\setbeamertemplate{footline}
{%
  \leavevmode%
  \hbox{%
  \begin{beamercolorbox}[wd=.92\paperwidth,ht=2.25ex,dp=1ex,center]{title in head/foot}%
  \end{beamercolorbox}%
  \begin{beamercolorbox}[wd=.08\paperwidth,ht=2.25ex,dp=1ex,leftskip=0cm plus1fill,rightskip=.015\paperwidth]{page number in head/foot}%
    \usebeamerfont{page number in head/foot}
    \insertframenumber{}
  \end{beamercolorbox}}%
  \vskip5pt%
}
\makeatother


% simple section page
\defbeamertemplate{section page}{simple}[1][]{
  \begin{centering}
    {\usebeamerfont{section name}\usebeamercolor[fg]{section name}#1}
    \vskip1em\par
    \begin{beamercolorbox}[sep=12pt,center]{part title}
      \usebeamerfont{section title}\insertsection\par
    \end{beamercolorbox}
  \end{centering}
}
\setbeamertemplate{section page}[simple]
\AtBeginSection{
  \frame[plain,c]{\sectionpage}
}

% mimic plain frame from metropolis theme
\newcommand{\plain}[2][]{%
  \begingroup
    \begin{frame}[plain,c]{#1}
      \begin{center}
        \bfseries\Large #2
      \end{center}
    \end{frame}
  \endgroup
}

% code colors
\colorlet{codeString}{Green}
\colorlet{codeKeyword}{RedOrange}

% workaround for problem with white text if notes are enabled
% http://tex.stackexchange.com/questions/232168/normal-text-is-invisible-when-using-beamer-with-notes-and-xelatex
\def\pgfsysdriver{pgfsys-dvipdfm.def}

\documentclass{beamer}

\usetheme{metropolis}
\setbeamercolor{normal text}{bg=white} % white background instead of black!2

% code colors from metropolis
\colorlet{codeString}{mLightGreen}
\colorlet{codeKeyword}{mLightBrown}


\usepackage{ifxetex}

\ifxetex
  \usepackage{polyglossia}
  \setmainlanguage{russian}
  \setotherlanguage{english}

  % workaround for "Package polyglossia Error: The current roman font does not contain the Cyrillic script!"
  \newfontfamily\cyrillicfonttt{Fira Mono}
\else
  \usepackage[T2A]{fontenc}
  \usepackage[utf8]{inputenc}
  \usepackage[english,russian]{babel}

  % workaround for "Package hyperref Warning: Glyph not defined in PD1 encoding"
  \hypersetup{unicode=true}
\fi

\newcommand{\eng}[1]{%
  \ifxetex%
    {\textenglish{#1}}%
  \else%
    {\foreignlanguage{english}{#1}}%
  \fi%
}


\usepackage{tikz}
\usetikzlibrary{arrows,matrix,positioning}

\usepackage{pgfpages}
%\setbeameroption{show notes on second screen}
\setbeamertemplate{note page}[plain]
\newcommand{\notep}[1]{\note{#1\par}}




\hypersetup{pdfauthor={Владимир Парфиненко}}
\title{Основы программирования}
\subtitle{Лекция № 3, 17 марта 2016 г.}
\date{}
\institute{
  \vspace{1em}
  \centering
  \parbox{0.9\textwidth}{
    \includegraphics[width=\linewidth]{xkcd_compiler_complaint}
    \par
    \raggedleft\tiny\url{http://xkcd.com/371}
  }
}

\begin{document}

% it can be done only after begin{document} because of "@"
\lstMakeShortInline[language=C,style=inline]@

\begin{frame}[plain]
  \titlepage
\end{frame}

\section{Массивы}

\begin{frame}[fragile]{Одномерные массивы}

  \notep{
    Мотивация: для задания квадратного уравнения нужно 3 переменные: a, b, c.
    А что делать с уравнением 10-ой степени?
    Правильно, нужен массив!
  }

  Массив~--- упорядоченный набор элементов одного типа.

  \pause
  Объявление массива @arr@, состоящего из @N@ элементов произвольного
  типа~@T@:
  \begin{clisting}
    T arr[N];
  \end{clisting}
  \revertListingParskip
  где @N@~--- константа времени компиляции.
  \notep{Без инициализации массив, объявленный в функции, содержит мусор.}

  \pause
  Пример объявления и инициализации массива:
  \begin{clisting}
    int xs[4] = {20, 10}; // {20, 10, 0, 0}
    int ys[]  = {20, 10}; // {20, 10}
  \end{clisting}
  \notep{Элементы, для которых нет явно заданного значения, инициализируются
  нулями.}
  \notep{Размер массива может быть вычислен автоматически, что может быть
  довольно удобным.}

\end{frame}

\begin{frame}[fragile]{Одномерные массивы: доступ к элементам}

  @T arr[N]@:
  \begin{center}
  \begin{tikzpicture}
    \matrix (m) [matrix of nodes, nodes in empty cells, column sep=-\pgflinewidth,
      row 1/.style={nodes={draw, minimum height=10mm, minimum width=10mm}},
      row 2/.style={nodes={align=center, text width=7mm}}
    ]{
          &     & &        & &       \\
      @0@ & @1@ & & \ldots & & @N-1@ \\
    };
  \end{tikzpicture}
  \end{center}
  \notep{Элементы массива размера N имеют индексы 0, 1, \ldots, N-1.}

  \pause
  Чтение элемента из массива по индексу @i@:
  \begin{clisting}
    T value = arr[i];
  \end{clisting}

  Запись элемента в массив по индексу @i@:
  \begin{clisting}
    arr[i] = new_value;
  \end{clisting}

  \notep{Доступ по недопустимому индексу~--- неопределенное поведение.}

\end{frame}


\section{Указатели}

\section{Виды памяти}

\section{Динамическая память}

\plain{Конец третьей лекции}

\end{document}
