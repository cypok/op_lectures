%\documentclass[xcolor=dvipsnames,12pt]{beamer}

% hide navigation symbols
\beamertemplatenavigationsymbolsempty

% add frame number
% TODO make it simplier
\makeatletter
\setbeamertemplate{footline}
{%
  \leavevmode%
  \hbox{%
  \begin{beamercolorbox}[wd=.92\paperwidth,ht=2.25ex,dp=1ex,center]{title in head/foot}%
  \end{beamercolorbox}%
  \begin{beamercolorbox}[wd=.08\paperwidth,ht=2.25ex,dp=1ex,leftskip=0cm plus1fill,rightskip=.015\paperwidth]{page number in head/foot}%
    \usebeamerfont{page number in head/foot}
    \insertframenumber{}
  \end{beamercolorbox}}%
  \vskip5pt%
}
\makeatother


% simple section page
\defbeamertemplate{section page}{simple}[1][]{
  \begin{centering}
    {\usebeamerfont{section name}\usebeamercolor[fg]{section name}#1}
    \vskip1em\par
    \begin{beamercolorbox}[sep=12pt,center]{part title}
      \usebeamerfont{section title}\insertsection\par
    \end{beamercolorbox}
  \end{centering}
}
\setbeamertemplate{section page}[simple]
\AtBeginSection{
  \frame[plain,c]{\sectionpage}
}

% mimic plain frame from metropolis theme
\newcommand{\plain}[2][]{%
  \begingroup
    \begin{frame}[plain,c]{#1}
      \begin{center}
        \bfseries\Large #2
      \end{center}
    \end{frame}
  \endgroup
}

% code colors
\colorlet{codeString}{Green}
\colorlet{codeKeyword}{RedOrange}

% workaround for problem with white text if notes are enabled
% http://tex.stackexchange.com/questions/232168/normal-text-is-invisible-when-using-beamer-with-notes-and-xelatex
\def\pgfsysdriver{pgfsys-dvipdfm.def}

\documentclass{beamer}

\usetheme{metropolis}
\setbeamercolor{normal text}{bg=white} % white background instead of black!2

% code colors from metropolis
\colorlet{codeString}{mLightGreen}
\colorlet{codeKeyword}{mLightBrown}


\usepackage{ifxetex}

\ifxetex
  \usepackage{polyglossia}
  \setmainlanguage{russian}
  \setotherlanguage{english}

  % workaround for "Package polyglossia Error: The current roman font does not contain the Cyrillic script!"
  \newfontfamily\cyrillicfonttt{Fira Mono}
\else
  \usepackage[T2A]{fontenc}
  \usepackage[utf8]{inputenc}
  \usepackage[english,russian]{babel}

  % workaround for "Package hyperref Warning: Glyph not defined in PD1 encoding"
  \hypersetup{unicode=true}
\fi

\newcommand{\eng}[1]{%
  \ifxetex%
    {\textenglish{#1}}%
  \else%
    {\foreignlanguage{english}{#1}}%
  \fi%
}

\usepackage{pgfpages}
%\setbeameroption{show notes on second screen}
\setbeamertemplate{note page}[plain]
\newcommand{\notep}[1]{\note{#1\par}}




\hypersetup{pdfauthor={Владимир Парфиненко}}
\title{Основы программирования}
\subtitle{Лекция № 3, 17 марта 2016 г.}
\date{}
\institute{
  \vspace{1em}
  \begin{center}
  \parbox{0.9\textwidth}{
    \includegraphics[width=\linewidth]{xkcd_compiler_complaint_ru}
    \par
    \raggedleft\tiny\url{http://xkcd.ru/371}
  }
  \end{center}
}

\begin{document}

% it can be done only after begin{document} because of "@"
\lstMakeShortInline[style=inlineC]@

\begin{frame}[plain]
  \titlepage
\end{frame}


\section{Массивы}

\begin{frame}[fragile]{Одномерные массивы}

  \notep{
    Мотивация: для задания квадратного уравнения нужно 3 переменные: a, b, c.
    А что делать с уравнением 10-ой степени?
    Правильно, нужен массив!
  }

  Массив~--- упорядоченный набор элементов одного типа.

  \pause
  Объявление массива @arr@, состоящего из @N@ элементов произвольного
  типа~@T@:
  \begin{clisting}
    T arr[N];
  \end{clisting}
  \revertListingParskip
  где @N@~--- константа времени компиляции.
  \notep{Без инициализации массив, объявленный в функции, содержит мусор.}

  \pause
  Пример объявления и инициализации массива:
  \begin{clisting}
    int xs[4] = {20, 10}; // {20, 10, 0, 0}
    int ys[]  = {20, 10}; // {20, 10}
  \end{clisting}
  \notep{Элементы, для которых нет явно заданного значения, инициализируются
    нулями.}
  \notep{Размер массива может быть вычислен автоматически, что может быть
    довольно удобным.}

\end{frame}

\tikzset{array matrix/.style={matrix of nodes, nodes in empty cells,
  column sep=-\pgflinewidth, row sep=-\pgflinewidth}}
\tikzset{array node/.style={draw, anchor=center, minimum height=9mm, minimum width=9mm}}
\tikzset{array index node/.style={align=center, anchor=center, text width=6mm}}

\begin{frame}[fragile]{Одномерные массивы: доступ к элементам}

  @T arr[N]@:
  \begin{center}
  \begin{tikzpicture}
    \matrix (m) [array matrix,
      row 1/.style={nodes={array node}},
      row 2/.style={nodes={array index node}}
    ]{
          &     & &        & &       \\
      @0@ & @1@ & & \ldots & & @N-1@ \\
    };
  \end{tikzpicture}
  \end{center}

  \pause
  Чтение элемента из массива по индексу @i@:
  \begin{clisting}
    T value = arr[i];
  \end{clisting}

  Запись элемента в массив по индексу @i@:
  \begin{clisting}
    arr[i] = new_value;
  \end{clisting}

  \notep{Доступ по недопустимому индексу (-1, N, 2*N)~--- неопределенное
    поведение.}

\end{frame}

\begin{frame}[fragile]{Двумерные массивы}

  @T arr[N][M]@:
  \begin{center}
  \begin{tikzpicture}[
    row2 columns/.style={row 2 column #1/.style={nodes={array node}}},
    row3 columns/.style={row 3 column #1/.style={nodes={array node}}},
    row4 columns/.style={row 4 column #1/.style={nodes={array node}}},
    row columns/.style={row#1 columns/.list={2,3,4,5,6}},
  ]
    \matrix (m) [array matrix,
      column 1/.style={nodes={array index node}},
      row 1/.style={nodes={array index node}},
      row columns/.list={2,3,4},
    ]{
      & @0@ & & \ldots & & @M-1@ \\
      @0@ & & & & & \\
      \vdots & & & & & \\
      @N-1@ & & & & & \\
    };
  \end{tikzpicture}
  \end{center}
  \notep{Первая размерность~--- количество строк, как в линейной алгебре.}

  \begin{columns}[onlytextwidth,T]
    \pause
    \begin{column}{0.5\textwidth}
      \begin{clisting}
        int m[2][3] = {
          {1, 2, 3},
          {4, 5, 6}};
      \end{clisting}
    \end{column}

    \pause
    \begin{column}{0.5\textwidth}
      \begin{clisting}
        int two = m[0][1];
        m[1][2] = 66;
      \end{clisting}
    \end{column}
    \notep{two получит значение 2, а вместо 6 будет записано 66.}

  \end{columns}

\end{frame}

\begin{frame}[fragile]{Многомерные массивы}

  @T arr[N1][N2][...][Nk]@~--- все аналогично.

\end{frame}

\begin{frame}[fragile]{Передача массивов в функции}

  Зачастую, размер массива передается отдельным параметром:
  \begin{clisting}[escapechar=\!]
    int sum_values(int values[], size_t size) !\pause!{
        int sum = 0;
        for (size_t i = 0; i < size; i++) {
            sum += values[i];
        }
        return sum;
    }
    !\pause!
    int sum_pairs(int pairs[][2], size_t size) {
        // ...
    }
  \end{clisting}

  \notep{size\_t~--- специальный беззнаковый тип данных, которого гарантированно
    хватает для хранения размера любого массива на данной архитектуре.}

  \notep{
    Для многомерных массивов значения всех размерностей, кроме первой
    фиксируются.
  }

\end{frame}

\begin{frame}[fragile]{Передача фиксированных массивов в функции}

  Массивы жестко фиксированных размерностей передаются без дополнительных
  параметров:
  \begin{clisting}
    int sum_matrix(int matrix[3][3]) {
        int sum = 0;
        for (size_t i = 0; i < 3; i++) {
            for (size_t j = 0; j < 3; j++) {
                sum += matrix[i][j];
            }
        }
        return sum;
    }
  \end{clisting}

\end{frame}

\begin{frame}[fragile]{Возвращение массивов из функций}

  \onslide<+->
  \begin{clisting}[escapechar=\!]
    int[] reverse(int arr[], size_t size) {
      !\onslide<+->{\comment// compile time error}!
        int result[size]; !\onslide<+->{\comment// compile time error}!
        // ...
        return result; !\onslide<+->{\comment// run time error}!
    }

    int original[] = {1, 2, 3};
    int reversed[] = reverse(original, 3);
      !\onslide<+->{\comment// compile time error}!
  \end{clisting}
  \notep{Формально говоря, функция в C вообще не может иметь массив в
    качестве возвращаемого значения.}
  \notep{size~--- не может быть использована для инициализации массива, т.\,к.
    не является константой времени компиляции.}
  \notep{Более того, такой массив уничтожится после возвращения из функции, но
    об этом позже.}
  \notep{Более того, в массивную переменную нельзя ничего <<присвоить>>, что
    делать с результатом?}

\end{frame}

\begin{frame}[fragile]{«Возвращение» массивов из функций}

  \begin{clisting}
    void reverse(int dst[], int src[], size_t size) {
        for (size_t i = 0; i < size; i++) {
            dst[i] = src[size - i - 1];
        }
    }

    int original[3] = {1, 2, 3};
    int reversed[3];
    reverse(reversed, original, 3);
    // use reversed
  \end{clisting}

  \notep{Перекладываем ответственность за создание массива на вызывающую
    функцию.}

\end{frame}


\section{Указатели}

\tikzset{
  byte lines/.style={black!30},
  block byte lines/.style={black!40},
  mem block/.style={fill=black!10},
}

\newcommand\memtop{1}
\newcommand\membottom{0}

\newcommand{\memline}[2]{
  \def\left{#1-0.45}
  \def\right{#2+0.45}

  \draw[byte lines] (\left,\membottom) grid (\right,\memtop);

  \foreach \y in {\membottom, \memtop}
    \draw (\left,\y) -- (\right,\y);

  \foreach \x in {\left, \right}
    \draw [decorate,decoration={snake,amplitude=0.3mm,segment length=3mm}]
      (\x,\membottom) -- (\x,\memtop);
}

\newcommand{\memblock}[3]{
  \def\left{#1}
  \def\right{#2}
  \def\blockname{#3}

  \draw [mem block,draw=none] (\left,\membottom) rectangle (\right,\memtop);
  \draw [block byte lines] (\left,\membottom) grid (\right,\memtop);
  \draw [mem block,fill=none] (\left,\membottom) rectangle (\right,\memtop)
    node [pos=0.5] (block \blockname) {};
}

\newcommand{\memaddr}[2]{
  \def\x{#1+0.5}
  \def\y{\membottom-0.5}
  \draw [-latex,shorten <=6pt] (\x,\y)
    node [fill=white,inner sep=0,outer sep=0] {\ttfamily\small #2} -- (\x,\membottom);
}

\newcommand{\memlabel}[3]{
  %\draw [decorate,decoration={brace, raise=0.7mm}] (#1,\memtop) -- (#2,\memtop)
  %  node [midway,yshift=4mm] {\ttfamily #3};
  \path (#1,\memtop) -- (#2,\memtop)
    node [midway,yshift=3mm,text height=1.5ex] {\ttfamily #3};
}

\newcommand{\memblockwithaddr}[5]{
  \memblock{#1}{#2}{#3}
  \memaddr{#1}{#4}
  \memaddr{#2-1}{#5}
}

\newcommand{\memblockwithaddrandlabel}[6]{
  \memblockwithaddr{#1}{#2}{#3}{#4}{#5}
  \memlabel{#1}{#2}{#6}
}

\newcommand{\memblockwithoneaddr}[4]{
  \memblock{#1}{#2}{#3}
  \memaddr{#1}{#4}
}

\newcommand{\memblockwithoneaddrandlabel}[5]{
  \memblockwithoneaddr{#1}{#2}{#3}{#4}
  \memlabel{#1}{#2}{#5}
}

\begin{frame}[fragile]{Указатель как адрес}

  \begin{center}
  \begin{visibleenv}<2->
  \begin{tikzpicture}[
      scale=0.8,
  ]

    \memline{1}{7}
    \memblock{2}{6}{x}
    \node [mem block] at (block x) {518};

    \memline{8}{14}
    \memblock{9}{13}{y}
    \node<-7> [mem block] at (block y) {320};
    \node<8-> [mem block] at (block y) {999};

    \begin{visibleenv}<3->
      \memaddr{2}{0x20}
      \memaddr{5}{0x23}

      \memaddr{9}{0x40}
      \memaddr{12}{0x43}
    \end{visibleenv}
  \end{tikzpicture}
  \end{visibleenv}
  \end{center}

  \begin{columns}[onlytextwidth,T]
    \begin{column}{0.25\textwidth}
      \begin{clisting}[escapechar=\!]
        int x = 518;
        int y = 320;
      \end{clisting}
    \end{column}

    \begin{column}{0.75\textwidth}<4->
      \begin{clisting}[escapechar=\!]
        int* ptr;
        !\onslide<5->!// !\comment\& - взятие адреса!
        // !\comment* - разыменование указателя!
        ptr = &x;
        printf("%p %d\n", ptr, *ptr); // 0x20 518
        !\onslide<6->!ptr = &y;
        printf("%p %d\n", ptr, *ptr); // 0x40 320
        !\onslide<7->!*ptr = 999;
        !\onslide<9->!printf("%d %d\n", x, y); // 518 999
      \end{clisting}
    \end{column}

  \end{columns}

  \notep{Адресом блока памяти считается номер его самого младшего байта.}
  \notep{int*~--- указатель на int, т.е. адрес памяти, где лежит int.}
  \notep{Заметьте, y поменяло свое значение, хотя прямых записей в него не было.}

\end{frame}

\begin{frame}[fragile]{Пример: возврат двух значений из функции}

  \begin{center}
  \begin{visibleenv}<3->
  \begin{tikzpicture}[
      scale=0.8,
  ]
    \memline{1}{7}
    \memblockwithaddrandlabel{2}{6}{q}%
      {0x20}{0x23}%
      {q}

    \memline{8}{14}
    \memblockwithaddrandlabel{9}{13}{r}%
      {0x40}{0x43}%
      {r}


    \node<4-8> [mem block] at (block q) {???};
    \node<9-> [mem block] at (block q) {12};

    \node<4-10> [mem block] at (block r) {???};
    \node<11-> [mem block] at (block r) {3};

  \end{tikzpicture}
  \end{visibleenv}
  \end{center}

  \begin{columns}[onlytextwidth,T]
    \begin{column}{0.5\textwidth}
      \begin{clisting}[escapechar=\!]
        void divide(
            int a, int b, !\onslide<6->{\comment// 123, 10}!
            int* pq, !\onslide<7->{\comment// 0x20}!
            int* pr) !\onslide<7->{\comment// 0x40}!
        {
          *pq = a / b; !\onslide<8->{\comment// *(0x20) = 12}!
          *pr = a % b; !\onslide<10->{\comment// *(0x40) = 3}!
        }
      \end{clisting}
    \end{column}

    \begin{column}{0.5\textwidth}<2->
      \begin{clisting}[escapechar=\!]
        int x = 123, y = 10;
        int q, r;

        !\onslide<5->!divide(x, y, &q, &r);
        !\onslide<12->!printf("%d = %d * %d + %d\n",
          x, y, q, r);
        // 123 = 10 * 12 + 3
      \end{clisting}
    \end{column}

  \end{columns}

  \notep{x, y передаются по значению, q, r передаются по указателю.}
  \notep{Модификация a и b в функции никак не изменит значения x, y.}
  \notep{Однако запись по указателям pq, pr непосредственно меняет значения
    q, r.}

\end{frame}

\begin{frame}[fragile]{Представление указателя}

  \begin{center}
  \begin{tikzpicture}[
      scale=0.7,
  ]
    \memline{1}{4}
    \memline{5}{11}
    \memline{12}{18}

    \memblockwithoneaddrandlabel{2}{3}{x}%
      {0x20}%
      {x}
    \node<-6> [mem block] at (block x) {37};
    \node<7-> [mem block] at (block x) {42};


    \begin{visibleenv}<2->
      \memblockwithaddrandlabel{6}{10}{px}%
        {0x80}{0x83}%
        {px}
      \node<3-> [mem block] at (block px) {\ttfamily 0x20};
    \end{visibleenv}

    \begin{visibleenv}<4->
      \memblockwithaddrandlabel{13}{17}{ppx}%
        {0xE0}{0xE4}%
        {ppx}
      \node<5-> [mem block] at (block ppx) {\ttfamily 0x80};
    \end{visibleenv}

  \end{tikzpicture}
  \end{center}

  \begin{clisting}[escapechar=\!]
    char x = 37;
    !\onslide<2->!char* px = &x;
    !\onslide<4->!char** ppx = &px;

    !\onslide<6->!**ppx = 42;
  \end{clisting}

  \notep{T* для любого T всегда занимает одинаковое количество байт на одной
    платформе. Обычно так: 32-битные платформы~--- 4 байта, 64-битные~--- 8
    байт.}
  \notep{Адрес хранится как беззнаковое число.}
  \notep{Сейчас не будем подробно останавливаться на двойных указателях.}

\end{frame}

\begin{frame}[fragile]{Представление массива}

  \begin{center}
  \begin{visibleenv}<2->
  \begin{tikzpicture}[
      scale=0.8,
  ]
    \memline{1}{9}
    \memblockwithoneaddrandlabel{2}{4}{a0}%
      {0x20}%
      {a[0]}
    \memblockwithoneaddrandlabel{4}{6}{a1}%
      {0x22}%
      {a[1]}
    \memblockwithoneaddrandlabel{6}{8}{a2}%
      {0x24}%
      {a[2]}

    \node [mem block] at (block a0) {10};
    \node<-5> [mem block] at (block a1) {20};
    \node<6-> [mem block] at (block a1) {200};
    \node<-7> [mem block] at (block a2) {30};
    \node<8-> [mem block] at (block a2) {230};

  \end{tikzpicture}
  \end{visibleenv}
  \end{center}

  \begin{columns}[onlytextwidth,T]
    \begin{column}{0.5\textwidth}
      \begin{clisting}[escapechar=\!]
        short a[3] = {10, 20, 30};

        !\onslide<3->!short* p = !\alt<12>{!a!}{!\&(a[0])!}!; // 0x20
        !\onslide<4->!short* p1 = p + 1;  // 0x22
        short* p2 = p + 2;  // 0x24
      \end{clisting}
    \end{column}

    \begin{column}{0.45\textwidth}
      \begin{clisting}[escapechar=\!]
        !\onslide<5->!!\temporal<9-10>{!*p1!}{!*(p+1)!}{!p[1]!}! = 200;
        !\onslide<7->!!\temporal<9-10>{!*p2!}{!*(p+2)!}{!p[2]!}! = !\temporal<9-10>{!*p1!}{!*(p+1)!}{!p[1]!}! + !\temporal<9-10>{!*p2!}{!*(p+2)!}{!p[2]!}!;
        !\onslide<10->!// *(x + y) == x[y]
      \end{clisting}
    \end{column}

  \end{columns}

  \notep{Массив хранится как непрерывный блок памяти, элементы располагаются
    последовательно.}
  \notep{При прибавлении значения n к указателю, его значение увеличивается на
    (n * размер типа данных под указателем).}
  \notep{Массивная переменная~--- всего лишь указатель на начало массива.}

\end{frame}

\begin{frame}[fragile]{Представление двумерного массива}

  \begin{center}
  \begin{visibleenv}<2->
  \begin{tikzpicture}[
      scale=0.8,
  ]
    \memline{1}{15}
    \memblockwithoneaddrandlabel{2}{4}{a00}%
      {0x20}%
      {\small a[0][0]}
    \memblockwithoneaddrandlabel{4}{6}{a01}%
      {0x22}%
      {\small a[0][1]}
    \memblockwithoneaddrandlabel{6}{8}{a02}%
      {0x24}%
      {\small a[0][2]}
    \memblockwithoneaddrandlabel{8}{10}{a10}%
      {0x26}%
      {\small a[1][0]}
    \memblockwithoneaddrandlabel{10}{12}{a11}%
      {0x28}%
      {\small a[1][1]}
    \memblockwithoneaddrandlabel{12}{14}{a12}%
      {0x2A}%
      {\small a[1][2]}

    \node [mem block] at (block a00) {10};
    \node [mem block] at (block a01) {20};
    \node [mem block] at (block a02) {30};
    \node [mem block] at (block a10) {40};
    \node [mem block] at (block a11) {50};
    \node [mem block] at (block a12) {60};

  \end{tikzpicture}
  \end{visibleenv}
  \end{center}

  \begin{columns}[onlytextwidth,T]
    \begin{column}{0.5\textwidth}
      \begin{clisting}[escapechar=\!]
        short a[2][3] = {
          {10, 20, 30},
          {40, 50, 60}};

        !\onslide<3->!a[i][j] == *((short*)a + i*3 + j);
      \end{clisting}
    \end{column}

    \begin{column}{0.45\textwidth}
      \begin{clisting}[escapechar=\!]
        !\onslide<4->!short* p = a;



        a[i][j] == p[i*3 + j];
      \end{clisting}
    \end{column}

  \end{columns}

  \notep{Массив хранится как непрерывный блок памяти, элементы располагаются
    последовательно: строка за строкой.}
  \notep{3~--- это ширина одной строки.}

\end{frame}


\section{Виды памяти}

\begin{frame}[fragile]{Статическая память}

  Примеры:
  \begin{itemize}
    \item глобальные переменные: @char game_field[100][100]@,
    \item строковые литералы: @"Hello world!"@,
    \item ...
      \notep{К статической памяти также относятся static-переменные.}
  \end{itemize}

  \pause
  Свойства:
  \begin{itemize}
    \item выделяется во время старта программы,
    \item освобождается во время завершения программы,
    \item инициализируется нулевыми значениями,
    \item фиксированный размер.
  \end{itemize}

\end{frame}

\begin{frame}[fragile]{Автоматическая память}

  Примеры:
  \begin{itemize}
    \item локальные переменные: @{...; int tmp; ...}@,
    \item аргументы функций: @(..., double speed, ...)@,
  \end{itemize}

  \pause
  Свойства:
  \begin{itemize}
    \item выделяется при входе в объемлющий блок,
    \item освобождается при выходе из объемлющего блока,
    \item изначально не инициализирована,
    \item фиксированный размер.
  \end{itemize}

\end{frame}

\begin{frame}[fragile]{Динамическая память}

  \notep{Недостатки автоматической и статической: фиксированный размер и
    недостаточно гибкое время жизни.}

  \pause
  Свойства:
  \begin{itemize}
    \item выделяется и освобождается по запросу программы,
    \item изначально не инициализирована,
    \item размер задается динамически.
  \end{itemize}

  \notep{Самый гибкий и самый сложный в работе вид памяти.}

\end{frame}

\section{Динамическая память}

\begin{frame}[fragile]{Библиотечные функции}

  Для работы с динамической памятью используется библиотека @stdlib.h@:
  \begin{itemize}
    \item @NULL@,
    \item @malloc@,
    \item @calloc@,
    \item @realloc@,
    \item @free@.
  \end{itemize}

  \notep{На слайдах мы будем опускать "\#include <stdlib.h>".}

\end{frame}

\begin{frame}[fragile]{Указатель, никуда не указывающий}

  @NULL@~--- специальное значение, символизирующее, что указатель не указывает
  ни на какой элемент памяти.
  \notep{Численное значение NULL == 0.}

  \begin{clisting}[escapechar=\!]
    int* ptr = NULL;
    !\onslide<2->!int value = *ptr; !\onslide<3->!// run time error
    !\onslide<2->!*ptr = 37; !\onslide<3->!// run time error
  \end{clisting}

  \onslide<3->
  Разные названия, но суть одна (segmentation fault, segfault, access
  violation, «Программа выполнила недопустимую операцию…», …).

\end{frame}

\begin{frame}[fragile]{Указатель, указывающий хоть куда}

  @void*@~--- специальный тип указателя, который может указывать на любой адрес
  в памяти. Может быть приведен к любого другому типу указателей и обратно.

  \begin{clisting}[escapechar=\!]
    int x = 37;
    int* px = &x;
    void* p = px;
    int* py = p;
  \end{clisting}

  \pause
  Минутка философии: \href{%
    http://thecodelesscode.com/case/5?lang=ru%
  }{%
    «Какова природа void?»~--- спросил учитель, …
  }
  \notep{Обязательно прочитать рассказ:\\http://thecodelesscode.com/case/5?lang=ru}

\end{frame}

\begin{frame}[fragile]{Выделение блока памяти}

  \begin{clisting}[basicstyle=\ttfamily]
    void* malloc(size_t size);
  \end{clisting}
  \revertListingParskip
  Функция выделяет блок памяти размером @size@ байт и возвращает указатель на
  начало блока. \alert<2>{В случае, если память выделить не получилось,
  возвращает \texttt{NULL}.}

  \onslide<3->
  \begin{clisting}[basicstyle=\ttfamily]
    void* calloc(size_t num, size_t size);
  \end{clisting}
  \revertListingParskip
  Функция выделяет блок памяти размером @num*size@ байт, зануляет его и
  возвращает указатель на начало. В случае, если память выделить не получилось,
  возвращает @NULL@.

  \notep{malloc = Memory ALLOCate}
  \notep{calloc = ALLOCate and Clear}

\end{frame}

\begin{frame}[fragile]{Освобождение блока памяти}

  \begin{clisting}[basicstyle=\ttfamily]
    void free(void* ptr);
  \end{clisting}

  Функция освобождает блок памяти. Если @ptr@ равен @NULL@, ничего не делает.

  \pause
  После вызова значение указателя @ptr@ остается прежним, но разыменовывать его
  нельзя.

  \pause
  Неиспользуемую память нужно обязательно освобождать, иначе рано или поздно
  она может кончиться (утечка памяти).

  \notep{Утечки памяти на серверных приложениях, которые должны работать
    несколько лет, недопустимы.}

  \notep{Утечки памяти в десктопных приложениях не так критичны, но могут и
    причинять неудобства (см. современные браузеры).}

\end{frame}

\begin{frame}[fragile]{Пример: массив динамического размера}

  \begin{center}
  \begin{tikzpicture}[
      scale=0.6,
  ]
    \memline{1}{23}

    \begin{visibleenv}<3-6>
      \memblock{2}{6}{p0}
      \memblock{6}{10}{p1}
      \memblock{10}{14}{p2}
      \memblock{14}{18}{p3}
      \memblock{18}{22}{p4}

      \node [mem block] at (block p1) {???};
      \node [mem block] at (block p3) {???};
    \end{visibleenv}

    \begin{visibleenv}<3->
      \memaddr{2}{0x100}
      \memaddr{21}{0x113}
    \end{visibleenv}

    \begin{visibleenv}<3-4>
      \node [mem block] at (block p0) {???};
      \node [mem block] at (block p2) {???};
      \node [mem block] at (block p4) {???};
    \end{visibleenv}
    \begin{visibleenv}<5-6>
      \node [mem block] at (block p0) {37};
      \node [mem block] at (block p2) {37};
      \node [mem block] at (block p4) {37};
    \end{visibleenv}

    \begin{visibleenv}<7->
      \node [fill=white] at (block p0) {???};
      \node [fill=white] at (block p1) {???};
      \node [fill=white] at (block p2) {???};
      \node [fill=white] at (block p3) {???};
      \node [fill=white] at (block p4) {???};
    \end{visibleenv}

  \end{tikzpicture}
  \end{center}

  \begin{clisting}[escapechar=\!]
    int n = 5;
    int* p; // !\comment\alt<3->{0x100}{???}!
    !\onslide<2->!p = malloc(n * sizeof(int));
    if (p == NULL) { /* error */ }
    !\onslide<4->!p[0] = p[n/2] = p[n-1] = 37;
    !\onslide<6->!free(p);
  \end{clisting}

  \notep{sizeof~--- оператор, выдающий размер типов данных. Например, на Intel
    x86 sizeof(int) == 4.}

\end{frame}

\begin{frame}[fragile]{Изменение размера блока памяти}

  \begin{clisting}[basicstyle=\ttfamily]
    void* realloc(void* ptr, size_t size);
  \end{clisting}

  Функция изменяет размер блока памяти до @size@ байт. В случае успешного
  изменения размера возвращает указатель на начало блока, иначе @NULL@.

  \pause
  Функция может как уменьшать размер, так и увеличивать. Возможно перемещение
  содержимого памяти, при этом возвращается указатель на новое
  месторасположение.

  \pause
  В случае неуспешного изменения размера, изначальный блок памяти не
  освобождается.

  \notep{Нельзя писать "ptr = realloc(ptr, ...)"~--- утечка памяти.}

\end{frame}

\begin{frame}[fragile]{Возвращение массивов из функций}

  \begin{clisting}
    int* reverse(int src[], size_t size) {
        int* dst = malloc(size * sizeof(int));
        if (dst == NULL) { return NULL; }

        for (size_t i = 0; i < size; i++) {
            dst[i] = src[size - i - 1];
        }
        return dst;
    }
    int original[3] = {1, 2, 3};
    int* reversed = reverse(original, 3);
    // use reversed
    free(reversed);
  \end{clisting}

  \notep{Теперь мы знаем, как мы можем вернуть массив из функции.}

\end{frame}

\begin{frame}[fragile]{Плоские динамические двумерные массивы}

  \begin{clisting}
    int* arr = malloc(n * m * sizeof(int));
    if (arr == NULL) { /* error */ }

    // arr[i*m + j] = 37;

    free(arr);
  \end{clisting}

  \notep{Плюсы: высокая производительность.}
  \notep{Минусы: дополнительная арифметика при доступе к элементам.}

\end{frame}

\begin{frame}[fragile]{Динамические массивы массивов: идея}

  \begin{center}
  \begin{tikzpicture}[
      scale=0.8,
  ]
    \memline{1}{11}
    \memblockwithoneaddrandlabel{2}{6}{a0}%
      {0x20}%
      {a[0]}
    \memblockwithoneaddrandlabel{6}{10}{a1}%
      {0x24}%
      {a[1]}

    \node [mem block] at (block a0) {\ttfamily 0x100};
    \node [mem block] at (block a1) {\ttfamily 0x200};


  \end{tikzpicture}

  \begin{tikzpicture}[
      scale=0.8,
  ]
    \memline{1}{9}
    \memblockwithoneaddrandlabel{2}{4}{a00}%
      {0x100}%
      {\small a[0][0]}
    \memblockwithoneaddrandlabel{4}{6}{a01}%
      {0x102}%
      {\small a[0][1]}
    \memblockwithoneaddrandlabel{6}{8}{a02}%
      {0x104}%
      {\small a[0][2]}

    \memline{10}{18}
    \memblockwithoneaddrandlabel{11}{13}{a10}%
      {0x200}%
      {\small a[1][0]}
    \memblockwithoneaddrandlabel{13}{15}{a11}%
      {0x202}%
      {\small a[1][1]}
    \memblockwithoneaddrandlabel{15}{17}{a12}%
      {0x204}%
      {\small a[1][2]}

    \node [mem block] at (block a00) {10};
    \node [mem block] at (block a01) {20};
    \node [mem block] at (block a02) {30};
    \node [mem block] at (block a10) {40};
    \node [mem block] at (block a11) {50};
    \node [mem block] at (block a12) {60};

  \end{tikzpicture}
  \end{center}

  \begin{clisting}[escapechar=\!]
    short** a = /* 0x20 */;
    // {{10, 20, 30}, {40, 50, 60}}

    a[i][j] == *(*(a + i) + j)
  \end{clisting}

  \notep{Храним массив указателей на другие массивы.}
  \notep{Плюсы: привычный синтаксис доступа к элементам.}
  \notep{Минусы: большее потребление памяти, лишняя косвенность.}

  \notep{Такой подход используется по умолчанию в некоторых языках
    программирования для реализации многомерных массивов. Например, в Java.}

\end{frame}

\begin{frame}[fragile]{Динамические двумерные массивы массивов}

  \begin{clisting}
    int** arr = malloc(n * sizeof(int*));
    if (arr == NULL) { /* error */ }
    for (int i = 0; i < n; i++) {
        arr[i] = malloc(m * sizeof(int));
        if (arr[i] == NULL) { /* error */ }
    }

    // arr[i][j] = 37;

    for (int i = 0; i < n; i++) {
        free(arr[i]);
    }
    free(arr);
  \end{clisting}

\end{frame}


\plain{Конец третьей лекции}

\end{document}

